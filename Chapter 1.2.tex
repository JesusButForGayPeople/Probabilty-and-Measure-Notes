\UNsection{1.2 - Probability Measures}
\seteqgroup{2}
\UNsubsection{Spaces and Fields} \quad 


Let $\Omega$ be an arbitrary \textbf{space} or set of points $\omega$. As it pertains to probability, $\Omega$ is the set of all possible outcomes $\omega$; a subset of $\Omega$ is an \textbf{event} and an element $\omega$ of $\Omega$ is a \textbf{sample point}. \\[10pt]
This definition is just too broad to do anything with in it's current form. Instead, we are interested in certain \textbf{classes} of sets that have special properties that allow us to actually accomplish something. A set that is part of a given class $\mathcal{F}$ is said to be \textbf{measurable} over $\mathcal{F}$. Specifically, if we want to systematize the kinds of problems presented in 1.1 we are interested in the class of sets that contains the intervals and is closed under the formation of countable unions and intersections.\\[10pt]

A collection of subsets $\mathcal{F}$ on $\Omega$ is called a \textbf{Semiring} if it satisfies the following three conditions:
\begin{enumerate}[label=\textbf{\roman*.}, topsep=0pt, itemsep=-3pt]
    \item $\emptyset \in \mathcal{F}$
    \item for any $A,B\in \mathcal{F}$, $A\cap B \in \mathcal{F}$
    \item for any $A,B\in \mathcal{F}$, $B \setminus A = \bigcup c_i$ where $c_i \in \mathcal{F}$ 
\end{enumerate}
\vspace{10pt}
A collection of subsets $\mathcal{F}$ on $\Omega$ is called a \textbf{Ring} if it satisfies the following three conditions:
\vspace{-10pt}
\[
\begin{split}
    \left.
    \begin{aligned}
        \textbf{i. } & \emptyset \in \mathcal{F}\\
        \textbf{ii. } & \text{for any }A,B\in \mathcal{F},\ A\cup B \in \mathcal{F}\\
        \textbf{iii. } & \text{for any }A,B\in \mathcal{F},\ B \setminus A \in \mathcal{F}\\
    \end{aligned}
    \right\}
\end{split}
\begin{split}
    \begin{aligned}
        & \text{ A Ring that also contains } \Omega \text{ is called a } \textbf{Field}
    \end{aligned}
\end{split}
\]

\vspace{10pt}

Out interest will primarily lie in considering fields, however it is important to clarify the subtle differences between semiring, rings, and fields.\\


\textbf{Field Axioms:}\\
A class $\mathcal{F}$ of an arbitrary nonempty space $\Omega$ is called a \textbf{Field} if it contains $\Omega$ itself and is closed under the formation of complements and finite unions. In otherwords, if it satisfies the first three of the following axioms:
\vspace{2ex}
\begin{enumerate}[label=\textbf{\roman*.}, topsep=0pt, itemsep=-3pt]
	\item $\Omega \in \mathcal{F}$ \textcolor{mygrey}{(Non-Empty)}
	\item $A \in \mathcal{F} \text{ implies } A^c \in \mathcal{F}$ \textcolor{mygrey}{(Closure Under Complementation)}
	\item $A,B \in \mathcal{F} \text{ implies } A \cup B \in \mathcal{F}$ \textcolor{mygrey}{(Closure Under \textbf{Finite} Union)}
	\item[] \elbowarrow  $\text{ implies } A \cap B \in \mathcal{F}$ \textcolor{mygrey}{(Closure Under \textbf{Finite} Intersection)}
	      \begin{tcolorbox}[
	      		colback=white,
	      		colframe=black,
	      		boxrule=0.4pt,
	      		sharp corners,
	      		right=2pt, top=2pt, bottom=2pt,
	      		enlarge left by=-1cm,left=2pt,
	      		width=\textwidth,
	      		enhanced,
	      	]
	      	{\item[] \textbf{iv.}\hspace{0.4em}$A_1, A_2, \dots \in \mathcal{F} \text{ implies } A_1 \cup A_2 \cup \cdots \in \mathcal{F}$} \textcolor{mygrey}{(Closure Under \textbf{Countable} Union)}\\[-10pt]
	      		\item[] \hspace{0.65cm} \elbowarrow  $\text{ implies } A_1 \cap A_2 \cap \cdots \in \mathcal{F}$ \textcolor{mygrey}{(Closure Under \textbf{Countable} Intersection)}
	      		\end{tcolorbox}
	      		\item[] \elbowarrow A field $\mathcal{F}$ that also satisfies \textbf{iv.} is called a \boldsymbol{$\sigma$}\textbf{-Field}. 
	      		\end{enumerate}
	      		\vspace{2ex}
	      		If $\mathcal{F}$ is closed under Complementation, then it is closed under the formation of finite unions if and only if it is closed under the formation of finite intersections. Furthermore, since all finite values are by definition countable, the definition of a $\sigma$-field is more restrictive than that of a field.\\[10pt]
	      		The largest $\sigma$-field in $\Omega$ is the \textbf{Power Class} $2^{\Omega}$, which consists of \textit{all} possible subsets of $\Omega$; conversely, the smallest $\sigma$-field consists only of the empty set and $\Omega$ itself.
                
	      		Let $\mathcal{A}$ be a class.\\[5pt]
	      		Consider the class of sets that...
	      		\vspace{-4.5ex}
	      		\[
	      			\hspace{-1cm}
	      			\begin{split}
	      				\begin{aligned}
	      					\text{i.}   & \text{ Contains } \mathcal{A}.      \\
	      					\text{ii.}  & \text{ Is a } \sigma \text{-field.} \\
	      					\text{iii.} & \text{ Is as small as possible.}    \\ \\
	      				\end{aligned}
	      			\end{split}
	      			\quad \Longrightarrow \quad
	      			\begin{split}
	      				\begin{gathered}\\ 
	      					\text{Such a class is the }\boldsymbol{\sigma}\textbf{-field }\\
	      					\textbf{generated by } \boldsymbol{\mathcal{A}}; \text{it is the } \\
	      					\text{intersection of all }\text{the } \sigma \text{-fields}\\
	      					  \text{ containing } \mathcal{A}. \text{ Denoted by } \sigma(\mathcal{A})\\
	      					\\
	      				\end{gathered}
	      			\end{split}
	      		\]
	      		
	      		\vspace{-2.5ex}
	      		
	      		When we say that $\sigma(\mathcal{A})$ is as small as possible we are saying that every $\sigma$-field that contains $\mathcal{A}$ also contains $\sigma(\mathcal{A})$.\\[5pt]

                \needspace{5\baselineskip}
                Stated more formally, $\sigma(\mathcal{A})$ has the following properties:
	      		\vspace{2ex}
	      		\begin{enumerate}[label=\textbf{\roman*.}, topsep=0pt, itemsep=-3pt]
	      			\item $\mathcal{A} \subset \sigma(\mathcal{A})$
	      			\item $\sigma(\mathcal{A})$ is a $\sigma$-field
	      			\item If $\mathcal{A} \subset \mathcal{G} $ and $\mathcal{G}$ is a $\sigma$-field, then $\sigma(\mathcal{A}) \subset \mathcal{G}$
	      		\end{enumerate}
	      		
	      		\UNsubsection{Probability Measures} \quad
	      		
	      		A \textbf{set function} is a real-valued function that is defined on some class of subsets of $\Omega$. A set function $P$ on a field $\mathcal{F}$ is a \textbf{Probability Measure} if it satisfies the following three conditions (Kolmogorov Axioms):
	      		\vspace{2ex}
	      		\begin{enumerate}[label=\textbf{\roman*.}, topsep=0pt, itemsep=-3pt]
	      			\item $0 \leq P(A) \leq 1$ for $A \in \mathcal{F}$; \quad  \textbf{\footnotesize Nonegativity of Probability}
	      			\item $P(\emptyset) = 0$, $P(\Omega) = 1$; \quad \textbf{\footnotesize Law of Total Probability}
	      			\item If $A_1, A_2, \dots$ is a disjoint sequence of $\mathcal{F}$-sets and if $\bigcup_{k=1}^\infty A_k \in \mathcal{F}$, then
	      			      \begin{UNequation}
	      			      	P\left( \bigcup_{k=1}^\infty A_k \right) = \sum_{k=1}^\infty P(A_k). \quad  \scalebox{0.8}[0.8]{\textbf{\footnotesize Finite Additivity of Probability}}
	      			      \end{UNequation}
	      		\end{enumerate}
	      		\vspace{2ex}
	      		
	      		If $\mathcal{F}$ is a $\sigma$-field in $\Omega$ and $P$ is a probability measure on $\mathcal{F}$, the triple $(\Omega, \mathcal{F}, P)$ is called a \textbf{Probability Measure Space} (or simply a probability space), and any $\mathcal{F}$-set $A$ for which $P(A)=1$ is called a \textbf{support} of $P$.\\[10pt]
    
                It is usually relatively easy to tell whether or not a given measure satisfies the first two conditions. In general, the hardest part of showing that a given $(\Omega, \mathcal{F}, P)$ is a probability measure space is showing that it satisfies finite additivity.\\ 
                
                \needspace{14\baselineskip}
                \textbf{Example 2.1:} 
                \begin{proofline}
                \[
                    \begin{aligned}
                    \textbf{Space: } & \text{Let } \Omega \text{ be a countable space.}\\
                    \boldsymbol{\sigma}\textbf{-field: } & \text{Let } \mathcal{F} \text{ be the $\sigma$-field of all subsets of } \Omega.\\
                    \textbf{Measure: } & \text{Let }  p(\omega) \text{ be a nonegative function on } \Omega \\
                                       & \text{such that } \sum_{\omega\in\Omega}p(\omega)=1. \text{ Then, define } P(A)= \sum_{\omega\in A}p(\omega).\\
                    \text{We let } {\textstyle A=\bigcup_{i=1}^{\infty}A_i},& \text{ where the } A_i \text{ are disjoint and } \omega_{i1}, \omega_{i2},... \text{ are the points in } A_i\\
                    \text{Then, by the definit} & \text{ion of a non-negative series, }\\
                    p(A)=\sum_{ij}p(\omega_{ij})&=\sum_i\sum_j p(\omega_{ij})= \sum_i P(A_i) \text{, thus } P \text{ is countably additive.}\\
                    \text{This } (\Omega, \mathcal{F}, P) \text{ is the }& \text{familiar } \textbf{Discrete Probability Space.}
                \end{aligned}
                \]
                \end{proofline}
                
	      		Suppose that $P$ is a probability measure on a field $\mathcal{F}$, and that $A,B \in \mathcal{F}$ and $A \subset B$. Since $P(A)+P(B-A)=P(B)$
	      		\begin{UNequation}
	      			P(A) \leq P(B) \quad  \quad  \text{if } A\subset B \quad \quad \textbf{\scriptsize Monotonicity}
	      		\end{UNequation}
	      		Furthermore, for any $A,B \in \mathcal{F}$
	      		\begin{UNequation}
	      			P(A\cup B)=P(A)+P(B)-P(A\cap B)
	      		\end{UNequation}
	      		Which is just the two variable case of the general \textbf{Inclusion-Exclusion Formula}:
	      		\begin{UNequation}
	      			P\left( \bigcup_{k=1}^{n} A_k \right)
	      			= \scalebox{0.75}[0.75]{$
	      				\substack{
	      					\begin{aligned}
	      						\sum_i P(A_i) & - \sum_{i<j} P(A_i \cap A_j) + \sum_{i<j<k} P(A_i \cap A_j \cap A_k)+\cdots \\
	      						              & \cdots + (-1)^{n+1} P(A_1 \cap \cdots \cap A_n)                             
	      					\end{aligned}}$}
	      		\end{UNequation}
	      		
	      		If $B_1 = A_1$ and $B_k = A_k \cap A_1^c \cap \cdots \cap A_{k-1}^c$, then the $B_k$ are disjoint and\\[5pt]
	      		$\bigcup_{k=1}^{n} A_k = \bigcup_{k=1}^{n} B_k$, so that
	      		$P\left( \bigcup_{k=1}^{n} A_k \right) = \sum_{k=1}^{n} P(B_k)$.
	      		Since $P(B_k) \leq P(A_k)$ by monotonicity:
	      		\begin{UNequation}
	      			P\left( \bigcup_{k=1}^{n} A_k \right) \leq \sum_{k=1}^{n} P(A_k). \quad \quad \textbf{\scriptsize Finite Subadditivity}
	      		\end{UNequation}

                \needspace{10\baselineskip}
	      		\textbf{Theorem 2.1: }Let $P$ be a probability measure on a field $\mathcal{F}$.
	      		\begin{enumerate}[label=\textbf{\roman*.}, topsep=0pt, itemsep=-3pt]
	      			\item \textbf{Continuity from below:} If $A_n$ and $A$ lie in $\mathcal{F}$ and $A_n \uparrow A$, then $P(A_n) \uparrow P(A).$
	      			\item \textbf{Continuity from above:} If $A_n$ and $A$ lie in $\mathcal{F}$ and $A_n \downarrow A$, then $P(A_n) \downarrow P(A)$.
	      			\item \textbf{Countable subadditivity:} If $A_1, A_2, \dots$ and $\bigcup_{k=1}^{\infty} A_k$ lie in $\mathcal{F}$ (the $A_k$ need not be disjoint), then
	      			      \begin{UNequation}
	      			      	P\left( \bigcup_{k=1}^{\infty} A_k \right) \leq \sum_{k=1}^{\infty} P(A_k).
	      			      \end{UNequation}
	      		\end{enumerate}
	      		\vspace{-6ex}
	      		\textbf{Proof:}
	      		\begin{proofline}
	      			\textbf{i:} Let $B_1=A_1$ and $B_k=A_k-A{k-1}$. Then, $B_k$ are disjoint, $A=\bigcup_{k=1}^{\infty}B_k$, and $A_n=\bigcup_{k=1}^{\infty}B_k$, thus by countable and finite additivity, $P(A)=\sum_{k=1}^{\infty}P(B_k)=\lim_n \sum_{k=1}^{n}P(B_k)=\lim_n P(A_n)$\\
	      			\textbf{ii.} $A_n \downarrow A$ implies $A_n^c \uparrow A^c$, meaning that $1-P(A_n)\uparrow 1 - P(A)$.\\
	      			\textbf{iii. } Using the definition finite subadditivity we can then apply the result from \textbf{i.} give the desired result. \hfill \qed 
	      		\end{proofline}
	      		
	      		\UNsubsection{Lebesgue Measure on the Unit Interval } \quad
	      		
	      		The current issue at hand is trying to prove that $P$ is countably additive. 
	      		For the sake of notational simplicity we will redefine $P$ to $\lambda$.\\[10pt]
	      		We begin by defining $\mathfrak{B}_0$ as a field of finite disjoint unions of intervals in $(0,1]$.
	      		From here we will define $\mathcal{J}$ to be the class of subintervals $(a,b]$ of $(0,1]$; then $\lambda(I)=|I|=b-a$ is ordinary length, and $\varnothing$ as a zero length element in $\mathcal{J}$. If $A=\cup_{i=1}^{n}I_i$, where $I_i$ are disjoint $\mathcal{J}$-sets. We can redefine how we assign probability to $A$ as:
	      		\begin{UNequation}
	      			\scalebox{0.7}[0.7]{$ \displaystyle P(A) = \sum_{i=1}^{n} |I_i| = \sum_{i=1}^{n} (b_i - a_i)$}
	      			\ \Longrightarrow \quad \lambda(A)=\sum_{i=1}^n\lambda(I_i)=\sum_{i=1}^n|I_i|
	      		\end{UNequation}
	      		
	      		Applying Theorem 1.3.iii to this new probability measure yields:
	      		\begin{UNequation}
	      			\sum_{i=1}^{n} |I_i| = \sum_{i=1}^{n} \sum_{j=1}^{m} |I_i \cap J_j| = \sum_{j=1}^{m} |J_j|
	      		\end{UNequation}
	      		
	      		Thus, (2.7) defines a set function $\lambda$ on $\mathcal{B_0}$ that we call the \textbf{Lebesgue Measure}.\\[10pt]
	      		
                \needspace{10\baselineskip}
	      		\textbf{Theorem 2.2: } The Lebesgue measure $\lambda$ is a countably additive probability measure on the field $\mathfrak{B}_0$.\\[10pt]
	      		\textbf{Proof: }
	      		\begin{proofline}
	      			Suppose that $A=\bigcup_{k=1}^{\infty}A_k$, where $A$ and the $A_k$ are $\mathfrak{B}_0$-sets and the $A_k$ are disjoint. Then $A=\bigcup_{i=1}^{n}I_i$ and $A_k=\bigcup_{j=1}^{m_k}J_{kj}$ are disjoint unions of $\mathcal{J}$-sets. Applying Theorem 1.3.iii to the definition of the Lebesgue Measure yields:
	      			\begin{UNequation}
	      				\begin{aligned}
	      					\lambda(A) 
	      					  & = \sum_{i=1}^{n} |I_i| = \sum_{i=1}^{n} \sum_{k=1}^{\infty} \sum_{j=1}^{m_k} |I_i \cap J_{kj}| \\
	      					  & = \sum_{k=1}^{\infty} \sum_{j=1}^{m_k} |J_{kj}| = \sum_{k=1}^{\infty} \lambda(A_k)             
	      				\end{aligned}
	      			\end{UNequation}
	      			
	      			\vspace{-4ex}
	      			\hfill \qed
	      		\end{proofline}
	      		  
