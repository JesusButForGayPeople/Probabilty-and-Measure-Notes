\UNsection{Simple Random Variables}
\seteqgroup{5}

Let $(\Omega, \mathcal{F}, P)$ be an arbitrary probability measure space, and let $X$ be a real-valued function on $\Omega$. $X$ is a \textbf{Simple Random Variable} if it has finitely many possible values, and if it satisfies:
\vspace{-1cm}
\begin{UNequation}
    \begin{aligned}
        &\hspace{3cm}
        [\omega:X(\omega)=x]\in \mathcal{F} \quad \scalebox{0.8}[0.8]{\text{for all }$x$}
    \end{aligned}
\end{UNequation}
\vspace{-1.67cm}
To actually decide if $X$\\
satisfies this condition we really only need to look at $\mathcal{F}$ rather than $P$, but before we can do that we have to ensure that $P[\omega:X(\omega)=x]$ is defined. Our previously discussed $d_n(\omega)$, the function that characterizes the digits of a dyadic expansion, is an example of a simple random variable  on the unit interval. For ease of notation it is very common in probability theory to drop the trailing $(\omega)$ in $X(\omega)$, as a result $X$ is used to denote the general value of $[w:X(\omega)=x]$. A finite sum of the form:

\begin{UNequation}
X=\sum_{i}x_{i}I_{A_{i }}    
\end{UNequation}

is a random variable if the $A_i$ form a finite partition of $\Omega$ into $\mathcal{F}$-sets. Furthermore, all simple random variables can be represented  as a finite sum of the above form. Simply take the range of $X$ and for each $x_i$ let $A_i=[X=x_i]$. However it is often more natural to replace $x_iI_{A_i}$ with $\sum_jx_i I_{A_{ij}}$ when the $A_{ij}$ form a finite decomposition of $A_i$ into $\mathcal{F}$-sets.

The $\sigma$-field $\sigma(X)$ generated by $X$ is the smallest $\sigma$-field with respect to which $X$ is measurable; that is $\sigma(X)$ is the intersection of all $\sigma$-fields with respect to which $X$ is measurable. For a sequence $X_1, X_2, ...$ of simple random variables , $\sigma(X_1, X_2, ...)$ is the smallest $\sigma$-field with respect to which each $X_i$ is measurable.

\vspace{1ex}

\textbf{Theorem 5.1: } Let $X_1,...,X_n$ be simple random variables.
\begin{enumerate}[label=\textbf{\roman*.}, topsep=1pt, itemsep=1ex]
    \item The $\sigma$-field $\sigma(X_1,...,X_n)$ consists of the sets:
    
    \vspace{-1.5em}
    
    \begin{UNequation}
        \textbf{[}\ (X_1,...,X_n)\in H\ \textbf{]}=  \textbf{[}\ \omega:(X_1(\omega),...,X_n(\omega)) \in H\ \textbf{]}
    \end{UNequation}

    \vspace{-2em}

    for $H\subset \mathbb{R}^n$; $H$ may be taken to be finite in this representation.
    \item A simple random variable $Y$ is measurable $\sigma(X_1,...,X_n)$ if and only if:

    \vspace{-1.5em}

    \begin{UNequation}
        Y=f(X_1,...,X_n) \quad \scalebox{0.8}[0.8]{\text{for some }} f:\mathbb{R}^n \rightarrow \mathbb{R}^1
    \end{UNequation}
\end{enumerate}

\textbf{Proof: }
\vspace{-1ex}
\begin{proofline}
    Let 
\end{proofline}