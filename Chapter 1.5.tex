\UNsection{1.5 - Simple Random Variables}
\seteqgroup{5}

Let $(\Omega, \mathcal{F}, P)$ be an arbitrary probability measure space, and let $X$ be a real-valued function on $\Omega$. $X$ is a \textbf{Simple Random Variable} if it has finitely many possible values, and if it satisfies:
\vspace{-0.75cm}
\begin{UNequation}
    \begin{aligned}
        &\hspace{3cm}
        [\omega:X(\omega)=x]\in \mathcal{F} \quad \scalebox{0.8}[0.8]{\text{for all }$x$}
    \end{aligned}
\end{UNequation}

\vspace{-6ex}
To actually decide if $X$\\
satisfies this condition we really only need to look at $\mathcal{F}$ rather than $P$, but before we can do that we have to ensure that $P[\omega:X(\omega)=x]$ is defined. Our previously discussed $d_n(\omega)$, the function that characterizes the digits of a dyadic expansion, is an example of a simple random variable  on the unit interval. For ease of notation it is very common in probability theory to drop the trailing $(\omega)$ in $X(\omega)$, as a result $X$ is used to denote the general value of $[w:X(\omega)=x]$. A finite sum of the form:

\begin{UNequation}
X=\sum_{i}x_{i}I_{A_{i }}    
\end{UNequation}

is a random variable if the $A_i$ form a finite partition of $\Omega$ into $\mathcal{F}$-sets. Furthermore, all simple random variables can be represented  as a finite sum of the above form. Simply take the range of $X$ and for each $x_i$ let $A_i=[X=x_i]$. However it is often more natural to replace $x_iI_{A_i}$ with $\sum_jx_i I_{A_{ij}}$ when the $A_{ij}$ form a finite decomposition of $A_i$ into $\mathcal{F}$-sets.

\vspace{2ex}

The $\sigma$-field $\sigma(X)$ generated by $X$ is the smallest $\sigma$-field with respect to which $X$ is measurable; that is $\sigma(X)$ is the intersection of all $\sigma$-fields with respect to which $X$ is measurable. For a sequence $X_1, X_2, ...$ of simple random variables , $\sigma(X_1, X_2, ...)$ is the smallest $\sigma$-field with respect to which each $X_i$ is measurable.

\vspace{2ex}

\needspace{6\baselineskip}
\textbf{Theorem 5.1: } Let $X_1,...,X_n$ be simple random variables.
\begin{enumerate}[label=\textbf{\roman*.}, topsep=1pt, itemsep=1ex]
    \item The $\sigma$-field $\sigma(X_1,...,X_n)$ consists of the sets:
    
    \vspace{-1.5em}
    
    \begin{UNequation}
        \textbf{[}\ (X_1,...,X_n)\in H\ \textbf{]}=  \textbf{[}\ \omega:(X_1(\omega),...,X_n(\omega)) \in H\ \textbf{]}
    \end{UNequation}

    \vspace{-2em}

    for $H\subset \mathbb{R}^n$; $H$ may be taken to be finite in this representation.
    \item A simple random variable $Y$ is measurable $\sigma(X_1,...,X_n)$ if and only if:

    \vspace{-1.5em}

    \begin{UNequation}
        Y=f(X_1,...,X_n) \quad \scalebox{0.8}[0.8]{\text{for some }} f:\mathbb{R}^n \rightarrow \mathbb{R}^1
    \end{UNequation}
\end{enumerate}

\vspace{-4ex}

\textbf{Proof: }
\vspace{-1ex}
\begin{proofline}
    Let $\mathcal{M}$ be the class of sets of the form (5.3):\\
    $\textbf{[}\ (X_1,...,X_n)\in H\ \textbf{]}=  \textbf{[}\ \omega:(X_1(\omega),...,X_n(\omega)) \in H\ \textbf{]}$. Sets of the form:\\
    $\textbf{[}(X_1, \ldots, X_n) = (x_1, \ldots, x_n)\textbf{]} = \bigcap_{i=1}^n \textbf{[}X_i = x_i\textbf{]}$ must line in $\sigma(X_1, \ldots, X_n)$; each set (5.3) is a finite union of sets of this form because $(X_1, \ldots, X_n)$, as a mapping from $\Omega$ to $\mathbb{R}^n$, has finite range. Thus $\mathcal{M} \subset \sigma(X_1, \ldots, X_n)$.

    On the other hand, $\mathcal{M}$ is a $\sigma$-field because:
    \begin{enumerate}[label=\textbf{\roman*.}, topsep=0pt, itemsep=0pt]
    \item $\Omega=\textbf{[}(X_1, \ldots, X_n) \in \mathbb{R}^n\textbf{]}$
    \item $\textbf{[}(X_1, \ldots, X_n) \in H\textbf{]}^c = \textbf{[}(X_1, \ldots, X_n) \in H^c\textbf{]},$
    \item $\bigcup_j \textbf{[}(X_1, \ldots, X_n) \in H_j\textbf{]} = \textbf{[}(X_1, \ldots, X_n) \in \bigcup_j H_j\textbf{]}.$
    \end{enumerate}
    But each $X_i$ is measurable with respect to $\mathcal{M}$, because $\textbf{[}X_i=x\textbf{]}$ can be put in the form of (5.3) by taking $H$ to consist of those $(x_1,...,x_n)$ in $\mathbb{R}^n$ for which $x_i=x$. It follows $\sigma(X_1,...,X_n)$ is contained in $\mathcal{M}$ and therefore equals $\mathcal{M}$. As intersecting $H$ with the (finite) range of $(X_1,...,X_n)$ in $\mathcal{R}^n$ does not affect (5.3), $H$ may be taken to be finite. This proves \textbf{i.}

    Assume that $Y$ has the form (5.4)—that is, $Y(\omega) = f(X_1(\omega), \ldots, X_n(\omega))$, for every $\omega$. Since $[Y = y]$ can be put in the form (5.3) by taking $H$ to consist of those $x = (x_1, \ldots, x_n)$ for which $f(x) = y$, it follows that $Y$ is measurable $\sigma(X_1, \ldots, X_n)$.

    Now assume that $Y$ is measurable $\sigma(X_1, \ldots, X_n)$. Let $y_1, \ldots, y_r$ be the distinct values $Y$ assumes. By part (i), there exist sets $H_1, \ldots, H_r$ in $\mathbb{R}^n$ such that:\\
    $\textbf{[}\omega : Y(\omega) = y_i] = [\omega : (X_1(\omega), \ldots, X_n(\omega)) \in H_i\textbf{]}$. Take $f = \sum_{i=1}^r y_i I_{H_i}$. Although the $H_i$ need not be disjoint, if $H_i$ and $H_j$ share a point of the form $(X_1(\omega), \ldots, X_n(\omega))$, then $Y(\omega) = y_i$ and $Y(\omega) = y_j$, which is impossible if $i \ne j$. Therefore each $(X_1(\omega), \ldots, X_n(\omega))$ lies in exactly one of the $H_i$, and it follows that $f(X_1(\omega), \ldots, X_n(\omega)) = Y(\omega)$. \hfill \qed
\end{proofline}

This theorem tells us that functions of simple random variables are themselves simple random variables; that is a $Y$ of the form (5.4) is  measurable $\sigma(X_1,...,X_n)$, it follows that $X^2, e^{tX},$ and so one are simple random variables.