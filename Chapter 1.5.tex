\UNsection{Simple Random Variables}
\seteqgroup{5}

Let $(\Omega, \mathcal{F}, P)$ be an arbitrary probability measure space, and let $X$ be a real-valued function on $\Omega$. $X$ is a \textbf{Simple Random Variable} if it has finitely many possible values, and if it satisfies:
\vspace{-1cm}
\begin{UNequation}
    \begin{aligned}
        &\hspace{3cm}
        [\omega:X(\omega)=x]\in \mathcal{F} \quad \scalebox{0.8}[0.8]{\text{for all }$x$}
    \end{aligned}
\end{UNequation}
\vspace{-1.67cm}
To actually decide if $X$\\
satisfies this condition we really only need to look at $\mathcal{F}$ rather than $P$, but before we can do that we have to ensure that $P[\omega:X(\omega)=x]$ is defined 